\chapter{Introduction}
\label{chap1}

\section{Theme}

This project concerns the efforts for decoupling web applications for the ATLAS, ALICE and LHCb experiments at CERN (European Organisation for Nuclear Research), resulting in separate front and backend modules. Thus, the problem is updating the previous in-house framework, moving away from its previous monolithic architecture, which led, over time, to a slow, unreliable, and unadaptable software development process with a high learning curve and a lack of defined standards. 

\section{Scope}

One of the multiple results of the collaboration between the Federal University of Rio de Janeiro (UFRJ) and CERN was the creation of Glance \cite{grael-tcc} in 2003. It was a technology which allowed members of the ATLAS experiment to create search interfaces on the web over multiple databases containing information about detector equipment and cables. It was later applied to other use cases, fetching different kinds of data from the experiments. With such relevance, the group of developers was named after Glance.

As Glance increased its maturity curve, new demands appeared for front-end features and more complex management of the resources with specific rules from each experiment. In 2013, the team started the development of FENCE (Front-End Engine for Glance) \cite{lange-tcc}: a web application framework designed to be used as the base for various systems at CERN. With it, the Glance Team developed 20 systems responsible for managing equipment, members, publications, authors, shifts, institutions, and many other essential resources for the major experiments at CERN. The developers realised that FENCE-based applications were hard to maintain due to their monolithic and rigid architecture, needing to adapt to new scenarios and include new technologies from a fast-evolving software industry.

This project started in January 2019, intending to expose publication data from the ATLAS Analysis system, developed and maintained by Glance Team, to the CERN Analysis Preservation group in a programmatic way via a REST API. The team quickly realised that such a solution could also be used as a starting point for breaking FENCE’s monolithic behaviour. In March 2019, a side project began to create a base structure for building APIs and decouple the front end from the rest of the source code.

\section{Justification}

After six years of intense usage of FENCE, it became an agreement between the Glance Team that the framework was the bottleneck of the development of the applications. One of the main issues of the current systems was the high coupling with FENCE. Every change in the framework would directly affect the multiple projects that depended on it, resulting in the risk of error propagation and increasing maintenance costs.

Another key problem with FENCE was that the user interface, business logic and data persistence code were also coupled, making it common to have them all on the same file. It made testing applications hard, making changes unpredictable and the software more unmaintainable.

Since most of the system's logic, including presentation, was executed on the server, pages were mostly static. Basic input validation, such as form pages, would require a request to the CERN servers. Collaborators from the experiments are geographically distributed, and the communication to distant machines increases the latency for the end users, affecting the application's usability. As the Glance Team is also dispersed in multiple countries, latency turned out to be an issue while debugging code, especially for remote developers, with multiple requests to the server and slowing the development process. Therefore, the segregation between front-end and back-end became a necessity to guarantee the survival of the applications in the upcoming years.

FENCE also depends on third-party libraries. However, the lack of tests makes the task of updating dependencies harder since a new version of a library may contain changes which affect projects in unexpected ways. Within years, all dependencies were outdated, including the PHP language, accumulating technical debt and security risks.

Therefore, such issues suggest the need to replace FENCE with another solution, segregating front-end and back-end layers as described in this project. Additionally, other aspects are considered in the solution, such as the high turnover of developers, independency of frameworks and tools, ease of changes on the code base to adapt to future software practices, documentation and composability.

The work done serves not only the developers of the Glance Team but also the 8800 collaborators of ATLAS, ALICE and LHCb who regularly use their applications. They can benefit from systems being built faster and more reliably, allowing the software to get new features at the same pace as the collaboration evolves.

\section{Goals}

The current project aims to create a standard structure for the Glance Team to develop REST APIs for building information systems for CERN experiments. Addressing the major pain points while using FENCE, the desired characteristics of the new software architecture were chosen: usability, interoperability, maintainability, extensibility, reusability and testability \cite{richards-architecture}. 

Therefore, the project comprises the following tasks:
\begin{itemize}
    \item Create a shared infrastructure to allow FENCE systems to expose REST API endpoints.
    \item Develop a new solution for building APIs, replacing FENCE and supporting authentication, authorisation, logging, e-groups and error handling.
    \item Propose a standard backend architecture for systems built by the Glance Team based on modern software industry standards.
\end{itemize}


\section{Methodology}

The work resulted from multiple side projects mixed with the need to share data with other groups and the development of modern web applications with the front end decoupled from the back end. Sharing data of systems already in production was the initial request and first step of the journey. New features were added to FENCE to facilitate the creation of query endpoints on top of the existing Object Relational Mapper (ORM) via JSON configuration files.

Afterwards, the challenge was to decouple the user interface from the business rules code, consume the new API to populate the browser pages with data, and send changes submitted via forms back to the server. By supporting write operations, APIs started to need to perform input validation and guard the domain rules invariants. FENCE was not designed for these use cases, and a study was done to propose a back-end structure that satisfied the architecture characteristics needs and relieved the major pain points caused by using the in-house framework.

When the team decided to move away completely from FENCE, the initial focus was feature parity of critical parts that were before delivered by the framework. Later, improvements were made, and new functionalities were added. The benefits and drawbacks of FENCE were kept in mind, with similar solutions for what worked well and changes to what caused struggles to the team while developing the back end.

All phases of the project followed the iterative process: after gathering the requirements, small solutions were built and released to be used by Glance developers. After, feedback was given on what could be added or improved on the solution, starting another development cycle. Quick iterations and interactions were possible because the project’s developers were part of the Glance team, working mostly in the same office. 

\section{Description}

The current project is divided into seven chapters. The current chapter corresponds to the introduction of the work.

\autoref{chap2} introduces other projects related to this work. Glance and FENCE are explained in more detail, the concept of exchanging data via REST APIs is presented, and the ALICE Service Work system is introduced.

In \autoref{chap3}, the development of the first version of the Frapi module is presented. It contains detailed information about FacTree and custom endpoints, together with other base features for creating REST APIs on FENCE applications.

\autoref{chap4} focuses on structuring the back end of the new systems by proposing a set of guidelines for developing complex applications and presenting the evolution of Frapi.

In \autoref{chap5}, the deprecation of Frapi and FENCE is explained. Alternative libraries to address the features of the deprecated framework are presented.

\autoref{chap6} lists the results of the current project, with statistics about packages and applications developed based on this work.

Finally, general conclusions and improvement suggestions are discussed on \autoref{chap7}.
