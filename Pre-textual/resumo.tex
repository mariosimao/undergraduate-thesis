\begin{abstract}

CERN é o maior laboratório de física de partículas do mundo, composto por diversos experimentos e contando com mais de 17.500 cientistas e engenheiros de 110 diferentes nacionalidades representando 580 universidades. Nesse contexto, surgiu a parceria com a UFRJ e o time Glance, responsável pelo desenvolvimento de aplicações web para a gerência de diferentes recursos dos experimentos. Este projeto tem como objetivo a criação de uma estrutura para facilitar o desenvolvimento de REST APIs em sistemas Glance com foco em interoperabilidade, manutenibilidade, extensabilidade e testabilidade. Primeiramente, foi criado o módulo Frapi para que sistemas construídos com \textit{framework} interno FENCE pudessem expor dados para outros grupos do CERN. Em seguida, foram estudadas arquiteturas e técnicas de desenvolvimentos para softwares com regras complexas a fim de propor um padrão na criação de APIs em sistemas Glance.
Este trabalho foi realizado presencialmente no CERN, em constante contato com coordenadores do experimento ALICE e engenheiros de diversos grupos do laboratório. As tecnologias desenvolvidas e padrões propostos resultaram na criação de dez sistema pelo time Glance expondo 546 \textit{endpoints} via REST APIs e possibilitaram num desenvolvimento mais rápido, confiável e com menos falhas.

\end{abstract}

