\begin{abstract}

CERN is the world's largest particle physics laboratory, comprising various experiments and involving over 17,500 scientists and engineers from 110 different nationalities representing 580 universities. In this context, a partnership was formed with UFRJ and the Glance team, responsible for developing web applications to manage different experiment resources. This project aims to create a framework to facilitate the development of REST APIs in Glance systems, focusing on interoperability, maintainability, extensibility, and testability. Initially, the Frapi module was created to allow systems built with the internal FENCE framework to expose data to other CERN groups. Subsequently, architectures and development techniques for software with complex rules were studied to propose a standard for API creation in Glance systems.
This work was carried out in person at CERN, in constant contact with ALICE experiment coordinators and engineers from various groups. The developed technologies and proposed standards resulted in the creation of ten systems by the Glance team, exposing 546 endpoints via REST APIs, enabling faster, more reliable, and less error-prone development.

\end{abstract}

